\documentclass{report}
\usepackage[utf8]{inputenc}

\title{Un breve acercamiento a la codificación en Python}
\author{Paul Maximiliano Rivera Medina\\
    \\
    Departamento de Física \\
    Universidad de Sonora} 

\date{23 enero de 2021} 

\usepackage{natbib}
\usepackage{graphicx}

\begin{document}

\maketitle

\section{Introduction}
Python en su aritmética básica cuenta con operadores sencillos para los números reales (adición, multiplicación y exponentes) expresiones lógicas (igualdades, comparaciones) y tres operadores lógicos.
Para esto es necesario la implementación de bibliotecas que nos brindan más herramientas para la elaboración de nuestros códigos.

NumPy es una biblioteca de Python que se utiliza para trabajar en el dominio de álgebra lineal, transformada de Fourier y matrices. NumPy fue creado en 2005 por Travis Oliphant. Es un proyecto de código abierto y puedes usarlo libremente.

Matplotlib. pyplot es una colección de funciones que hacen que matplotlib funcione como MATLAB. Cada función de pyplot realiza algún cambio en una figura: por ejemplo, crea una figura, crea un área de trazado en una figura, traza algunas líneas en un área de trazado, decora la trama con etiquetas, etc.


\section{Antecedentes}
Python es un lenguaje de programación multiparadigma, ya que soporta orientación a objetos, programación imperativa y, en menor medida, programación funcional. Es un lenguaje interpretado, dinámico y multiplataforma.

Google Colab es un servicio cloud, basado en los Notebooks de Jupyter, que permite el uso gratuito de las GPUs y TPUs de Google, con librerías como: Scikit-learn, PyTorch, TensorFlow, Keras y OpenCV. Todo ello con bajo Python 2.7 y 3.6.

LaTeX es un sistema de composición de textos, orientado a la creación de documentos escritos que presenten una alta calidad tipográfica. Por sus características y posibilidades, es usado de forma especialmente intensa en la generación de artículos y libros científicos que incluyen, entre otros elementos, expresiones matemáticas.

\section{Retroalimentación}
\begin{enumerate}
\item \textbf{¿Qué te pareció?}\\
\textit{muy larga, no en general fue mas que nada la percepción ya que estuve muy al pendiente de las clases el miércoles y jueves pensé que solo era las primeras dos actividades, eran las que estaban el martes y al hacer el reporte me entere que estaban otras dos}
\item \textbf{¿Cómo estuvo la carga de trabajo?}\\
\textit{pues no estuvo mal la cantidad, pero siento que es mas fácil para uno administrarse  cuando ya sabe que es todo lo que tiene que hacer. Pero esta en todo su derecho de hacerlo, al menos así empezare (no se si soy el único) a  hacer todo a primera hora}
\item \textbf{¿Qué se te dificultó más?} \\
\textit{leer el documento de la tercer actividad, como era mucho texto y la verdad siendo un método que no creo que vaya a usar, sumado a otros pendientes me resulto difícil el empezar a leerlo, creo que es la única dificultad}
\item \textbf{¿Qué te aburrió?}\\
\textit{la tercer actividad}
\item \textbf{¿Qué recomendarías para mejorar la primera Actividad?}\\
\textit{cuidar la redacción, a todos nos falla, fuera de eso no veo como mejorarla estaba muy completa}
\item \textbf{¿Que grado de complejidad le asignarías a esta Actividad? (Bajo, Intermedio, Avanzado)} \\
\textit{bajo, pensé que era mas al ver el tamaño de la pagina web}

\end{enumerate}

\section{Conclusión}
En esta actividad se solicitó la elaboración de cuatro ejercicios, los cuales se podrían sintetizar a cada uno en: implementación de matemática básica, uso de procesos lógicos, manejo de loops, exploración de comandos para gráficos. Las primeras dos actividades fueron sencillas, son cosas hechas en cursos anteriores con un código distinto. El hacerlas con anterioridad ayuda mucho, en este caso lo nuevo es la combinación dos códigos, python para el desarrollo y látex para las notas me resulta muy agradable, ya que deja presentable el código. La tercera actividad me costó un poco el acostumbrarme a la forma de codificar los comandos lógicos, errores de sintaxis. La última actividad, si bien, tarde tiempo por el “prueba y error” fue muy divertido el jugar con las distintas opciones que brinda la biblioteca matplotlib
\end{document}
