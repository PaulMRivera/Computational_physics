\documentclass{article}

%caracteres en español ASCII extendido
\usepackage[utf8]{inputenc}
%caracteres en español ASCII extendido

%introducir imágenes
\usepackage{graphicx}
%introducir imágenes

%poner hipervínculos
\usepackage{hyperref}
\hypersetup{
    colorlinks=true,
    linkcolor=blue,
    filecolor=magenta,      
    urlcolor=cyan,
}

\urlstyle{same}
%poner hipervínculos

%para poner las lineas de la portada
\newcommand{\HRule}[1]{\rule{\linewidth}{#1}}
%para poner las lineas de la portada

\begin{document}

\begin{titlepage}

\title{ \normalsize 
		%\begin{figure}
        \begin{center}
        \includegraphics[height=6cm]{Logo.jpg}
        \end{center}
       % \end{figure}
        \LARGE \textsc{\textbf{Universidad De Sonora}} \\ \bigskip
		\Large División de Ciencias Exactas y Naturales \\
        Licenciatura en Física \\ \bigskip
        \bigskip
       Física Computacional
		\\ [0.1cm]  
		\HRule{2pt} \\
		\Large \textbf{{Reporte de Actividad 7}} \\
        \textit{\textbf{"Álgebra Lineal con Python"}}
		\HRule{2pt} \\
		\normalsize \vspace*{0.001\baselineskip}}
        
\date{\bigskip \Large Hermosillo, Sonora  \hspace*{\fill}  5 de Marzo de 2021}

\author{
		\Large\textbf{Paul Maximiliano Rivera Medina} \\ \bigskip
        \\ \bigskip
       \Large Profr. Carlos Lizárraga Celaya}
       \end{titlepage}
       \maketitle

\newpage

\section{\LARGE Introducción}
En esta actividad trabajamos con el tema de álgebra lineal. Repasamos lo visto anteriormente en los cursos de álgebra lineal ahora implementando problemas computacionales con Python en la actividad de número 7. Hicimos uso de nuevas bibliotecas en Python, para poder realizar los ejercicios computacionales de nuestro nuevo tema para poder poner en práctica la teoría vista durante toda la semana. Es por eso que en esta sección trabajamos con los temas de matrices, determinantes, matrices cuadradas, el teorema de Caley-Hamilton, ecuaciones características, polinomios, resolución de sistema de ecuaciones, aplicaciones del Gauss-Jordan y eliminación Gaussiana, soluciones entre otras funciones elementales del álgebra lineal.
\large\textbf{}  

 \section{\LARGE Colusión}
 Podemos ver que es fundamental el álgebra lineal en la física computacional, ya que con esta podemos modelar muchas situaciones de la vida real y poder jugar con los datos de formas en las que podamos usar los datos y sus predicciones a nuestro favor. Es por eso que haber realizado esta actividad me gusto mucho por que se relaciona con lo que me quiero de dedicar terminando la universidad, a al ciencia de datos, a lo que se relaciona todo esto de la estadística, el álgebra lineal y la física computacional.
 \large\textbf{}
 
 \section{\LARGE Apéndice }
 
 \begin{enumerate}
     \item\large\textbf{¿Qué te pareció?}
     \\Me gusto mucho el volver a usar matrices.
 \\
 \\
    \item\large\textbf{¿Cómo estuvo estuvo la carga de trabajo?}
    \\Ligera comparada con las cinco y seis.
 \\
 \\
    \item\large\textbf{¿Qué se te dificultó más?} 
    \\ Empezar elaborar la actividad.
 \\
 \\
    \item\large\textbf{¿Qué te aburrió?}
    \\En esta actividad nada.
 \\
 \\

    \item\large\textbf{¿Que grado de complejidad le asignarías a esta Actividad?}
    \\Facil, estuvo muy sencillo, las notas estaban muy rápida
 \end{enumerate}
 

\end{document}

